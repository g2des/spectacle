\documentclass{article}


\usepackage{arxiv}

\usepackage[utf8]{inputenc} % allow utf-8 input
\usepackage[T1]{fontenc}    % use 8-bit T1 fonts
\usepackage{hyperref}       % hyperlinks
\usepackage{url}            % simple URL typesetting
\usepackage{booktabs}       % professional-quality tables
\usepackage{amsfonts}       % blackboard math symbols
\usepackage{nicefrac}       % compact symbols for 1/2, etc.
\usepackage{microtype}      % microtypography
\usepackage{lipsum}

\title{A Systematic Approach to Analyzing Syndication of Viral News Stories}


\author{
 Gaurav Deshpande \\
  Institute for Software Research\\
  Carnegie Mellon University\\
  Pittsburgh, PA 15213 \\
  \texttt{gdeshpan@andrew.cmu.edu} \\
  %% examples of more authors
 \And
    Sam E. Teplov\\
    %\thanks{Use footnote for providing further
    %information about author (webpage, alternative
    %address)---\emph{not} for acknowledging funding agencies.} \\
  Institute for Software Research\\
  Carnegie Mellon University\\
  Pittsburgh, PA 15213 \\
  \texttt{steplov@andrew.cmu.edu} \\
 \And
   Alon Peer \\
  Department of Computer Science\\
  Carnegie Mellon University\\
  Pittsburgh, PA 15213 \\
  \texttt{apeer@andrew.cmu.edu} \\
 \AND
    Dr. Nicolas Christin \\
    Department of Computer Science \\
    Carnegie Mellon University\\
    Pittsburgh, PA 15213 \\
   \texttt{nicolasc@andrew.cmu.edu} \\
  %% \And
  %% Coauthor \\
  %% Affiliation \\
  %% Address \\
  %% \texttt{email} \\
  %% \And
  %% Coauthor \\
  %% Affiliation \\
  %% Address \\
  %% \texttt{email} \\
}

\begin{document}
\maketitle

\begin{abstract}
Working on the abstract 
\end{abstract}


% keywords can be removed
\keywords{Viral news \and News propagation \and News Syndication}


\section{Introduction}


\section{Backgrgound}
%\label{sec:headings}


\section{Related Work}
\cite{zannettou2018origins}

\section{Methodology}

\section{Analysis}

\section{Findings}

\section{Future Work}

\section{Conclusion}

%\label{sec:others}
%\lipsum[8] \cite{kour2014real,kour2014fast} and see \cite{hadash2018estimate}.

%The documentation for \verb+natbib+ may be found at
%\begin{center}
%  \url{http://mirrors.ctan.org/macros/latex/contrib/natbib/natnotes.pdf}
%\end{center}
%Of note is the command \verb+\citet+, which produces citations
%appropriate for use in inline text.  For example,
%\begin{verbatim}
%   \citet{hasselmo} investigated\dots
%\end{verbatim}
%produces
%\begin{quote}
%  Hasselmo, et al.\ (1995) investigated\dots
%\end{quote}

%\begin{center}
%  \url{https://www.ctan.org/pkg/booktabs}
%\end{center}


%\subsection{Figures}
%\lipsum[10] 
%See Figure \ref{fig:fig1}. Here is how you add footnotes. \footnote{Sample of the first footnote.}
%\lipsum[11] 

%\begin{figure}
 % \centering
 % \fbox{\rule[-.5cm]{4cm}{4cm} \rule[-.5cm]{4cm}{0cm}}
 % \caption{Sample figure caption.}
  %\label{fig:fig1}
%\end{figure}

%\subsection{Tables}
%\lipsum[12]
%See awesome Table~\ref{tab:table}.

%\begin{table}
 %\caption{Sample table title}
  %\centering
  %\begin{tabular}{lll}
    %\toprule
   % \multicolumn{2}{c}{Part}                   \\
    %\cmidrule(r){1-2}
   % Name     & Description     & Size ($\mu$m) \\
   % \midrule
   % Dendrite & Input terminal  & $\sim$100     \\
   % Axon     & Output terminal & $\sim$10      \\
    %Soma     & Cell body       & up to $10^6$  \\
    %\bottomrule
  %\end{tabular}
  %\label{tab:table}
%\end{table}

%\subsection{Lists}
%\begin{itemize}
%\item Lorem ipsum dolor sit amet
%\item consectetur adipiscing elit. 
%\item Aliquam dignissim blandit est, in dictum tortor gravida eget. In ac rutrum magna.
%\end{itemize}


\bibliographystyle{unsrt}  
\bibliography{paper}  %%% Remove comment to use the external .bib file (using bibtex).
%%% and comment out the ``thebibliography'' section.


%%% Comment out this section when you \bibliography{references} is enabled.



\end{document}